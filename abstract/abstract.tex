\documentclass[a4paper,11pt]{article}
\usepackage[utf8]{inputenc}

\title{An integrated formulation for merging mixture components based on posterior probabilities}
\author{Marc Comas-Cufí \and Josep-Antoni Martín-Fernández \and Glòria Mateu-Figueras}

\begin{document}
\maketitle

In parametric clustering, groups are usually formed from the components of a finite mixture distribution. The method comprises two steps: first, a finite mixture distribution with  the following probability density function
\[
f(\;\cdot\; ; \pi_1, \dots, \pi_k, \theta_1 \dots \theta_k) = \pi_1  f (\;\cdot\; ; \theta_1) + \dots + \pi_k f(\;\cdot\; ; \theta_k)
\]
is fitted to a data set; second, each observation $x$ is assigned to the component $j$, $1\leq j \leq k$, with $\hat{\pi}_j f( x ; \hat{\theta}_j)$ maximum. After the fitting process, some mixture components might not be separated enough. To deal with this situation, several authors have proposed merging methods that, based on the posterior probabilities incrementally, combine those components that are more similar. 

Using a generic definition, an integrated formulation to unify such merging methods is presented and discussed. This new formulation opens the way to define new methods based on the posterior probabilities.

%
%  BAUDRY, J.-P., RAFTERY A.~E., CELEUXx, G., LO, K., and GOTTARDO, R. (2010). Combining Mixture Components for Clustering. {\em Journal of Computational and Graphical Statistics}, 9(2):332-353.
%
%
%  HENNIG, C. (2010): Methods for merging Gaussian mixture components. {\em Advances in Data Analysis and Classification}, 4(1):3--34.\\
%
%
% LONGFORD, N.~T. and BARTOSOVA, J. (2014). A confusion index for measuring separation and clustering. {\em Statistical Modelling}, 14(3):229--255.
%

\end{document}


