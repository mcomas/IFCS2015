\documentclass[a4paper,11pt]{article}
\usepackage[utf8]{inputenc}

\title{An integrated formulation for merging mixture components based on posterior probabilities}
\author{Marc Comas-Cufí \and Josep-Antoni Martín-Fernández \and Glòria Mateu-Figueras}

\begin{document}
\maketitle

In parametric clustering, groups are usually formed from the components of a finite mixture distribution. The method consists of two steps: first, a finite mixture distribution with probability density function
\[
f(\;\cdot\; ; \pi_1, \dots, \pi_k, \theta_1 \dots \theta_k) = \pi_1  f (\;\cdot\; ; \theta_1) + \dots + \pi_k f(\;\cdot\; ; \theta_k)
\]
is fitted to the data set.  In a second step, each observation $x$ is assigned to the component $j$, $1\leq j \leq k$, with $\hat{\pi}_j f( x ; \hat{\theta}_j)$ maximum. After the fitting process, some mixture components might not be separated enough. 

To deal with this situation, several authors have proposed merging methods based on the posterior probabilities. These methods use  the information contained in the posterior probabilities to incrementally combine those components which are more similar. 

In this presentation, we introduce an integrated formulation for unifying the different approaches. In addition, this new formulation opens the way for the definition of new approaches.

\end{document}


